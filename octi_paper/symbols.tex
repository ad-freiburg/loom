\documentclass{article}
\usepackage{amsmath}
\usepackage{array}

\newcommand\todo[1]{\textcolor{blue}{[TODO: #1]}}
\newcommand\TODO[1]{\textcolor{blue}{\small [TODO: #1]}}

\newenvironment{eqdesc}
  {\par\vspace{\abovedisplayskip}\noindent\begin{tabular}{>{$}l<{$} @{${}\hspace{0.6cm}{}$} l}}
  {\end{tabular}\par\vspace{\belowdisplayskip}}

% Input Graph

\def\iG{G}  % Input graph
\def\iV{V}  % Input nodes
\def\iv{v}  % Input node
\def\iE{E}  % Input edges
\def\ie{e}  % Input edge

% Drawing

\newcommand\drawing[1]{\mathcal{D}_{#1}}  % Drawing of graph
\newcommand\initdrawing[1]{\drawing{#1}^*}  % Drawing of graph

\def\drawingcurvesym{c}  % edge curve part of drawing
\def\drawingpossym{p}    % node position part of drawing

\newcommand\drawingcurve[1]{\drawingcurvesym(#1)}  % Drawing curve of argument edge
\newcommand\drawingpos[1]{\drawingpossym(#1)}  % Drawing pos urve of argument node

% Grid

\def\gScale{C}
\def\gmind{\hat d}


% Octilinear grid graph

\def\gG{\Gamma}  % Grid graph
\def\gV{\Psi}  % Grid graph nodes
\def\gv{\psi}  % Grid graph node
\def\gE{\Omega}  % v edges
\def\ge{\omega}  % Grid graph edge

\newcommand\ggv[2]{\gv_{#1, #2}}  % Grid node at #1, #2
\newcommand\gpv[3]{\gv_{#1, #2}^{#3}}  % Port #3 node at #1, #2
\newcommand\gse[3]{\ge_{#1, #2}^{#3}}  % Sink edge #3 node at #1, #2

\def\gPath{p}    % path on grid graph


\newcommand\gPturn[1]{c_{#1}}  % Turn penalty
\def\gHopcost{c_h}    % Hop cost in grid graph
\newcommand\gPcost[1]{c(#1)}  % Path cost

\begin{document}

\section{Symbology of ``Metro Maps on Octilinear Grid Graphs''}

\subsection{General}

\begin{eqdesc}
	\iG = (\iV, \iE)     &  Input graph with nodes $\iv \in \iV$ and edges $\ie = (u, v) \in \iE$ \\
	\drawing{\iG} = (\drawingpossym, \drawingcurvesym)     &  Drawing of $\iG$, $\drawingpos{\iv}$ assigns a position to every $\iv \in \iV$, $\drawingcurve{e}$ \\
	 & assigns a curve to every $\ie \in \iE$. \\   
	\initdrawing{\iG} &  The initial embedding of $\iG$. \\
	L^{\infty} & Uniform norm. \\
	\gScale & scale factor of our grid, essentially the grid cell dimension
	\gmind & min distance between edges in octilinear drawing
\end{eqdesc}

\subsection{Grid Graph}

\begin{eqdesc}
	\gG  = (\gV, \gE)     &  Octilinear grid graph with nodes $\gv \in \gV$ and edges $\ge \in \gE$.\\
	\gPath = (\gv_0, \gv_1, ..., \gv_n)  & path in $\gG$.\\
	\ggv{x}{y} & Original grid node at position $(x, y)$.
\end{eqdesc}


\end{document}
