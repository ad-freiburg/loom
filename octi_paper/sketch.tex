\documentclass{sig-alternate-sigmod09}

\usepackage[bookmarks=true,pdfborder= 0 0 0]{hyperref}

\usepackage{tikz}
\usetikzlibrary{calc,trees,positioning,arrows,chains,shapes.geometric,%
  decorations.pathreplacing,decorations.pathmorphing,shapes,%
  matrix,shapes.symbols,plotmarks,decorations.markings,shadows}

\DeclareMathOperator{\atantwo}{atan2}

\hypersetup{
pdfauthor={Patrick Brosi},
pdfkeywords=,
pdftitle={An Approximation Algorithm for Metro Map Drawing},
pdfsubject={},
pdfcreator={},
pdfproducer={}
}

\begin{document}
\title{An Approximation Algorithm for Metro Map Drawing}

\numberofauthors{1}
%\author{Patrick Brosi\\\affaddr{University of Freiburg}\\\affaddr{Chair of Algorithms and Data Structures}}

\maketitle

\section{Abstract}

We investigate a novel approximative approach to the NP-hard problem of octilinear Metro Map drawing. Contrary to previous work, which usually relied on either local or global optimization techniques, we state the task as a (greedy) iterative shortest-path problem on a specially crafted octilinear grid graph, whose edge weights are updated after each iteration. While our results are not perfect, they come surprisingly close to previous work which used Integer Linear Programming to find a globally optimal solution. We state the basic idea of our approach, give some heuristics to improve the final result and evaluate our method on 10 cities around the world. As far as we are aware, our approach is the first non-global approach which guarantees octilinear results, albeit at the cost of not always finding a solution. The resulting schematic maps are rendered using previous work by us and are publicly accessible at http://bla.blubb. 

\section{Introduction}

Lorem ipsum 

\section{Problem definition}

Given an undirected planar graph $G = \{V, E\}$. We say $\mathcal{D}_G = \{p, c\}$ is a drawing of $G$, where $p(v) \in \mathbb{R}^2$ assigns a position to every node $v \in V$ and $c(e) = (q_0, q_1, ..., q_n)$, $q_i \in \mathbb{R}^2$ assigns a piecewise linear curve to every edge $e \in E$. Our goal is to find a schematic drawing of $G$ that resembles a classic Metro Map. This is usually formalized as a set of hard and soft constraints \cite{nb, ...}. The hard constraints may be summarized as:

\begin{enumerate}
\setlength\itemsep{.1em}
\item \emph{Octilinearity}. Each edge curve $c(e)$ may only consist of segments whose orientation is a multiple of $45^{\circ}$.
\item \emph{Topology preservation}. The input embedding should be respected. In particular, no new crossings between edges should be introduced and non-incident edges should never share common points. This is often modelled as a minimum distance $d_{e}$ between non-incident edge curves and a minimum distance $d_{v}$ between node positions.
\end{enumerate}

Additionally, the following soft constraints are usually employed:

\begin{enumerate}
\setlength\itemsep{.1em}
\item \emph{Edge monotony}. Minimize the number of turns an edge has to take. Prefer large angles.
\item \emph{Geographical accuracy}. The original node positions should be distorted as little as possible.
\item \emph{} Balance the angular distribution of incident edges in nodes.
\item \emph{Map density}.
\end{enumerate}

This list of soft constraints does not claim to be complete. % some more info on additional constraints

\section{Related Work}

Survey Nöllenburg %http://i11www.iti.kit.edu/extra/publications/n-asamm-14.pdf
Survey Wolff %http://www1.pub.informatik.uni-wuerzburg.de/pub/wolff/pub/w-dsms-07.pdf
Hong et al., force-based approach
Stott's PhD
steiner trees! %https://www.researchgate.net/profile/Matthias_Mueller-Hannemann/publication/225160153_Approximation_of_Octilinear_Steiner_Trees_Constrained_by_Hard_and_Soft_Obstacles/links/0912f50cf248ec1193000000/Approximation-of-Octilinear-Steiner-Trees-Constrained-by-Hard-and-Soft-Obstacles.pdf

\section{Map Generation}

To find an octilinear drawing for some graph $G$, our algorithm proceeds in three basic steps: 1. Build a so-called octilinear grid graph $\Omega$ on which the drawing of a shortest path between two nodes is guaranteed to be octilinear and as smooth as possible. 2.) For an unsettled edge $e = \{u, v\}$ in $G$ find suitable nodes $u'$ and $v'$ in $\Omega$ and search for the shortest path from $u'$ to $v'$. 3. Based on the path found for $e$ in step 2, balance $\Omega$ locally. Mark $e$ as settled and greedily continue with step 2.

This section gives detailed descriptions of all steps involved. Section~\ref{SEC:} describes additional balancing constraints to preserve the original embedding and discusses heuristics to prevent dead-locks.

\subsection{Octilinear Grid Graph}

We start by constructing an auxiliary directed graph $\Omega = \{V_\omega, E_\omega\}$ with a drawing $\mathcal{D}_\Omega = \{p_\omega, c_\omega\}$ in such a way that every possible path $(e_0, e_1, ..., e_n), e_i \in E_\Omega$ is automatically represented as an octilinear curve by concatenating $c_\omega(e_0), c_\omega(e_0),...,c_\omega(e_n)$. The graph in Fig.~\ref{FIG:gridgraph} trivially satisfies this: we simply define a $n\times m$ grid, add nodes $v_{n,m}$ to $V_\omega$ for every grid point and set $p_\omega(v_{n,m}) = (n, m)$. Each node $v_{n,m}$ is connected with its 8 direct neighbors (except at the grid boundaries, where some neighbors are not present).

% TODO: labelling!
% We relate to these neighbors by their relative position, beginning with 0 (``north'' or $0^{\circ}$) and continuing clock-wise. The edges are labeled accordingly. So, $e_{n,m}^0 = (v_{n,m}, v_{n,m}^0) = (v_{n,m}, v_{n, m+1})$, $e_{n,m}^1 = (v_{n,m}, v_{n,1}^1) = (v_{n,m}, v_{n+1,m+1})$, and so on (Fig. TODO). For each such edge $e_{n, m}^i = (v_{n, m}, v_{n, m}^i), i \in [0, 7]$, we simply set $c_\omega(e_{n, m}^i) = \left(p_\omega\left(v_{n, m}\right), p_\omega\left(v_{n, m}^i\right)\right)$. Since for each edge $(v_{n, m}, v_{n, m}^i)$ there will be an opposite edge $(v_{n, m}^, v_{n, m})$, $\Omega$ is effectively an undirected graph. Edge directions are thus not indicated in Figures ?? and ??.

To optimize soft constraint (1), we additionally want the total cost for a path from some node $v \in V_\omega$ to $u \in V_\omega$ to also reflect the total number of necessary turns. The cost of a turn should be weighted by its degree - either $135^{\circ}$, $90^{\circ}$ or $45^{\circ}$. We call these penalties $p_{135}$, $p_{90}$ and $p_{45}$. A straight pass through a node should go unpunished, that is, $p_{180} = 0$. Since we aim for a ``smooth'' path through $G$ and want to favor obtuse angles, we require $p_{180} < p_{135} < p_{90} < p_{45}$. The smoothest octilinear path from $u$ to $v$ in $\Omega$ would then be just the shortest path between them.

\begin{figure}
  \centering
	$\vcenter{\hbox{\includegraphics[width=0.474\textwidth]{figures/grid.pdf}}}$
	\caption{Left: A shortest path between $t$ and $u$ on a grid graph with uniform edge cost. Path turns are not minimized. Right: Two shortest path between $(t, u)$, and $(v, w)$ on our octilinear grid graph, $(t, u)$ acts as an obstacle for $(v, w)$. Path turns are minimized.}
	\label{FIG:paths}
\end{figure}

We model this by adding 8 \emph{port} nodes $v_{n,m}^{0} ... v_{n,m}^{7}$ to every node $v_{n,m}$. Each port again corresponds to an outgoing direction and is connected with a direct edge $e_{n,m}^0 ... e_{n,m}^7$. We make sure that the cost for each of these edges bigger than the maximum cost of any other edge. This ensures that these edges are only used if we actually want to arrive in $v_{n,m}$. For paths passing through $v_{n,m}$, we connect each port with its sibling ports at $90^{\circ}$ $135^{\circ}$ and $180^{\circ}$ (Fig.~\ref{FIG:port}, middle).

A $180^{\circ}$ pass through $v_i$ can now take the direct edge connecting each port $v_i^k$ with its $180^{\circ}$ sibling $v_i^{(k + 4) \mod 8}$ (Fig.~\ref{FIG:paths}, 1). Equally, a $90^{\circ}$ degree turn in $v_i$, coming from $v_i^k$ can take the edge to $v_i^{(k + 2) \mod 8}$ (Fig.~\ref{FIG:paths}, 2) and so on.

As both a $45^{\circ}$ edge and a $90^{\circ}$ edge may be replaced by two (or more) cheaper edges, special care has to be applied to the modelling of the actual edge costs. A $45^{\circ}$ turn can be simulated by first passing $v_i$ on a $180^{\circ}$ edge, and then again on a $135^{\circ}$ edge (Fig.~\ref{FIG:paths}, 3). As $p_{180} < p_{135} < p_{45}$, this path may be cheaper than $p_{45}$. Similarily, a $90^{\circ}$ degree turn can be simulated by two cheaper $135^{\circ}$ edges (Fig.~\ref{FIG:paths}, 4).

We call the actual edge costs for the 4 classes of port-to-port edges $c_0$, $c_{135}$, $c_{90}$ and $c_{45}$. To prevent the shortcuts described above, we make use of the fact that the absolute costs of these edges are irrelevant - since they are modeled the same in every node in $\Omega$ and a path passing through (not going to) some $v_{n,m}$ has to take one of them, we just have to make sure that the relative costs reflect the penalties we want to apply to the different angles, that is, $c_{135} - c_{180} = p_{135}$, $c_{90} - c_{180} = p_{90} = p_{135} + (c_{90}-c_{135})$ and so on. 

\begin{figure}[h]
  \centering
	$\vcenter{\hbox{\includegraphics[width=0.45\textwidth]{figures/node.pdf}}}$
	\caption{A $3\times3$ grid graph. Each node $v_i$ has 8 ports $v_i^0 ... v_i^7$ which are connected to $v_i$ by a direct edge. Each port is additionally connected to its $180^{\circ}$, $135^{\circ}$ and $35^{\circ}$ neighbor ports.}
	\label{FIG:gridgraph}
\end{figure}

Following this insight, we introduce a constant $a \geq 0$ and set the inter-port edge costs as follows:
\begin{align}
c_{180} &= a + p_{180} = a \\
c_{135} &= a + p_{135} \\
c_{90} &= a + p_{90} \\
c_{45} &= a + p_{45}.
\end{align}

We choose $a$ in a way such that the following inequalities are fullfilled:
\begin{align}
2a + p_{135} &\geq a + p_{90} \label{CONSTRS:sim90}\\
2a + p_{135} + p_{90} &\geq a + p_{45}\label{CONSTRS:sim45}.
\end{align}
(\ref{CONSTRS:sim90}) ensures that simulating a $90^{\circ}$ pass with two $135^{\circ}$ passes is never cheaper than $c_{90}$. (\ref{CONSTRS:sim45}) ensures that simulating a $45^{\circ}$ pass with a $135^{\circ}$ pass and a $180^{\circ}$ pass is never cheaper than $c_{90}$.

(\ref{CONSTRS:sim45}) and (\ref{CONSTRS:sim90}) are fullfilled for $a = p_{45} - p_{135} \geq 0$, leading to the following edge costs in $\Omega$:

\begin{align}
c_{180} &= p_{45} - p_{135} \\
c_{135} &= p_{45} \\
c_{90} &= p_{45} - p_{135} + p_{90} = c_{180} + p_{90} \\
c_{45} &= 2 p_{45} - p_{135} = c_{180} + c_{135}.
\end{align}

We can thus omit explicit edges for $45^{\circ}$ turns, as they can always be exactly replaced by a $180^{\circ}$ and a $135^{\circ}$ edge.


\begin{figure*}
  \centering
	$\vcenter{\hbox{\includegraphics[width=0.9\textwidth]{figures/paths.pdf}}}$
	\caption{1. A $180^{\circ}$ pass through a node $v$. 2. A $90^{\circ}$ pass through $v$. 3. A $45^{\circ}$ pass through $v$ simulated by a $180^{\circ}$ and $135^{\circ}$ pass. 4. A $90^{\circ}$ pass through $v$ simulated by two $135^{\circ}$ passes. }
	\label{FIG:paths}
\end{figure*}

\subsection{Shortest-Path Iteration}

\subsection{Graph Balancing}

\subsection{Preserving Topology}

\subsection{Input Ordering}

\section{Evaluation}

\subsection{Penalty Experiments}

\section{Conclusion}

\balancecolumns
\end{document}
